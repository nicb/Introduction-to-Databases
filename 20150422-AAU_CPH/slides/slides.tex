%
% $Id: slides.tex 145 2015-04-22 09:22:42Z nicb $
%
% Copyright (C) 2015 Nicola Bernardini nicb@sme-ccppd.org
% 
% This work is licensed under a Creative Commons License, and specifically the
% 
%   Creative Commons Attribution-ShareAlike 2.5 License
%   http://creativecommons.org/licenses/by-sa/2.5/legalcode
% 
% Check http://www.creativecommons.org/ for more information on
% Creative Commons Licenses and the Creative Commons Project.
%
% Set the macros below to whatever is appropriate in a given context
%

\newcommand{\imagedir}{../../images}
\newcommand{\exampledir}{../../examples}
\newcommand{\templatedir}{../../../latex/beamer/sme-ccppd}
\documentclass[\printmode,compress,xcolor=dvipsnames]{beamer}

\usetheme{Copenhagen}
\usecolortheme{crane}

\usepackage{colortbl}
\usepackage[british]{babel}
\usepackage{pgf}
\usepackage{xspace}
\usepackage{verbatim}

\usepackage{multimedia}
\usepackage{xmpmulti}
\usepackage{hyperref}
\usepackage[nofancy]{svninfo}
\newcommand{\rcstag}{v.\svnInfoRevision\ \ \svnInfoDate\xspace}
\usepackage{gensymb}

\newcommand{\hhref}[1]{\href{#1}{#1}}

\newcommand{\cpyear}{2015}
\newcommand{\cpholder}{Nicola Bernardini}
\newcommand{\cpholderemail}{n.bernardini@conservatoriosantacecilia.it}

% Use some nice templates

\beamertemplateshadingbackground{red!10}{structure!10}
\beamertemplatetransparentcovereddynamic
\beamertemplateballitem
\beamertemplatenumberedballsectiontoc

% My colors
\definecolor{notdone}{gray}{0.35}

%\beamertemplateshadingbackground{white!10}{white!10}

\title[Databases]
{%
  A (short) Introduction to Databases\\
  {\tiny (\rcstag)}
}

\author{%
  Nicola Bernardini\\
    \href{mailto:\cpholderemail}{\cpholderemail}
}
\institute[SMERM]%
{%
  \href{http://www.conservatoriosantacecilia.it}
  {``S.Cecilia'' Conservatory - Rome}
}
\date[Copenhagen, 22/04/2015]{\scriptsize April 22 2015\\Service Systems Design Master\\Aalborg University in Copenhagen\\Copenhagen, Denmark}

\begin{document}
\svnInfo $Id: slides.tex 145 2015-04-22 09:22:42Z nicb $
\newcounter{ms}
  
\begin{frame}
  \titlepage
\end{frame}

\section{Introduction}

\newlength{\leftcolumn}
\setlength{\leftcolumn}{0.45\textwidth}
\newlength{\rightcolumn}
\setlength{\rightcolumn}{0.55\textwidth}

\begin{frame}
  \frametitle<+->{Topics we will cover}

  \begin{itemize}[<+- | alert@+->]

      \item Just a touch of history

			\item Why do we need databases

			\item What are they, anyway?

			\item How do they work (in a very basic form)

			\item Exercising database design

  \end{itemize}

\end{frame}

\begin{frame}
				\frametitle<+->{Topics we will \emph{NOT} cover}

	\uncover<+->{\alert{(but feel free to ask about them if you feel so inclined, by any means :-))}}

  \begin{itemize}[<+- | alert@+->]

      \item Advanced topics, such as:

  				\begin{itemize}[<+- | alert@+->]

						\item search algorithms

						\item data indexing

						\item complex multi--table queries

						\item sophisticated database functions

						\item \dots

          \end{itemize}

  \end{itemize}

\end{frame}

\pgfdeclareimage[width=\leftcolumn]{leibnitz computer}{\imagedir/calcmach}

\setcounter{ms}{0}
\refstepcounter{ms}
\begin{frame}
  \frametitle<+->{What are computers for? (\arabic{ms})}

  \begin{columns}[T]
    \begin{column}{\leftcolumn}
      \begin{figure}[H!]
        \pgfuseimage<5->{leibnitz computer}
				\uncover<5->{\caption{A Leibnitz computing machine}}
      \end{figure}
    \end{column}
    \begin{column}{\rightcolumn}
      \begin{itemize}[<+- | alert@+->]

          \item Back in the old days (I mean the eighteenth century)\dots

          \item \dots computers were conceived and built for calculus and
                  computation

          \item Mathematicians found out that reality was quite more
                complicate than the abstract models they had conceived\dots

          \item \dots so they invented ``computing technologies'', that is
                  \uncover<+->{\bfseries computers}
                  

      \end{itemize}
    \end{column}
  \end{columns}
\end{frame}

\pgfdeclareimage[width=\leftcolumn]{IBM 7094}{\imagedir/IBM7094}
\refstepcounter{ms}
\begin{frame}
  \frametitle<+->{What are computers for? (\arabic{ms})}

  \begin{columns}[T]
    \begin{column}{\leftcolumn}
      \begin{figure}[H!]
				\pgfuseimage<5->{IBM 7094}
				\uncover<5->{\caption{An IBM 7094 (1962)}}
      \end{figure}
    \end{column}
    \begin{column}{\rightcolumn}
      \begin{itemize}[<+- | alert@+->]

          \item The last time computers were used \emph{solely} for computing purposes
                  was during World War II

          \item At the end of WW II, computing technology was mature enough to
                enter the industrial world\dots

          \item \dots and indeed, the computing capability of computers faded into the
                  background

      \end{itemize}
    \end{column}
  \end{columns}
\end{frame}

\pgfdeclareimage[width=\leftcolumn]{samsung galaxy}{\imagedir/samsung_galaxy}

\refstepcounter{ms}
\begin{frame}
  \frametitle<+->{What are computers for? (\arabic{ms})}
  \begin{columns}[T]
    \begin{column}{\leftcolumn}
      \begin{figure}[H!]
          \pgfuseimage<5->{samsung galaxy}
      \end{figure}
    \end{column}
    \begin{column}{\rightcolumn}
      \begin{itemize}[<+- | alert@+->]

          \item Today, the most useful function of computers is\dots

          \item \dots their ability to \emph{\bfseries store and organize data}

          \item and this is where \uncover<+->{\bfseries databases kick in.}

          \item Computation is still used of course, but it is no longer the
                core function \uncover<+->{(it appears in several specific
                fields in what are now called ``number crunching
                applications'')}

      \end{itemize}
    \end{column}
  \end{columns}
\end{frame}

\section{Databases}

\subsection{Introduction}

\pgfdeclareimage[width=\leftcolumn]{data in ram}{\imagedir/data_in_ram}

\setcounter{ms}{0}
\refstepcounter{ms}
\begin{frame}
  \frametitle<+->{So why do we need databases? (\arabic{ms})}

  \begin{columns}[T]
    \begin{column}{\leftcolumn}
			\uncover<4->{\verbatiminput{\exampledir/int_42.c}}
      \pgfuseimage<4->{data in ram}
    \end{column}
    \begin{column}{\rightcolumn}
      \begin{itemize}[<+- | alert@+->]

         \item As you all know, computers structure (and store) information in RAM memory

         \item Data maps into memory straight memory

         \item Memory is characterized by a content and an address (pointer)

         \item An address can also be considered some form of content

      \end{itemize}
    \end{column}
  \end{columns}
\end{frame}

\pgfdeclareimage[width=\leftcolumn]{data struct in ram}{\imagedir/data_struct_in_ram}

\refstepcounter{ms}
\begin{frame}
  \frametitle<+->{So why do we need databases? (\arabic{ms})}

  \begin{columns}[T]
    \begin{column}{\leftcolumn}
			\uncover<4->{\verbatiminput{\exampledir/birthday_ds.c}}
      \pgfuseimage<4->{data struct in ram}
    \end{column}
    \begin{column}{\rightcolumn}
      \begin{itemize}[<+- | alert@+->]

         \item Furthermore, data in memory can be structured

         \item That is, it can be organized so that complex structures
               are kept physically together in RAM memory

         \item In the past few decades, these structures have become
                 \emph{object}, in object--oriented programming lingo

      \end{itemize}
    \end{column}
  \end{columns}
\end{frame}

\refstepcounter{ms}
\begin{frame}
  \frametitle<+->{So why do we need databases? (\arabic{ms})}

  \begin{itemize}[<+- | alert@+->]

     \item However, data in RAM memory is not persistent

     \item When the computer shuts off, the data is gone

     \item So: data needs to be \emph{persisted}

     \item Which translates into: \uncover<+->{\bfseries databases}

  \end{itemize}

\end{frame}

\subsection{Types}

\setcounter{ms}{0}
\refstepcounter{ms}
\begin{frame}
  \frametitle<+->{How many different kinds of databases are there?}

  \begin{itemize}[<+- | alert@+->]

    \item You can make databases (that is: save your data) with just about
          anything

          \begin{itemize}[<+- | alert@+->]

            \item log files

            \item line oriented delimited text\\
                    (f. ex. {\texttt Nicola;Bernardini;Rome;14/08/1956;+4512345678})

            \item spreadsheets

            \item disk file systems

            \item proper relational database

          \end{itemize}

    \item only the last two do not save only \emph{data} but also \emph{relations}

  \end{itemize}

\end{frame}

\subsection{Relations}

\setcounter{ms}{0}
\refstepcounter{ms}
\begin{frame}
  \frametitle<+->{What are relations for? (\arabic{ms})}

  \begin{itemize}[<+- | alert@+->]

    \item If you start collecting data you soon realize that

      \begin{enumerate}[<+- | alert@+->][a) ]

          \item either you repeat parts of the data endlessly

          \item or you have to save \emph{data relations} along with the \emph{data} itself

      \end{enumerate}

    \item and here we get to the {\bfseries golden rule of databases}, namely\dots

  \end{itemize}

\end{frame}

\begin{frame}
  \frametitle<+->{The Golden Rule of Databases}

  \begin{center}
    \begin{Huge}
      \begin{bfseries}
        \alert{DO NOT REPLICATE DATA.}
      \end{bfseries}
    \end{Huge}
  \end{center}

\end{frame}

\begin{frame}
  \frametitle<+->{Rationale for the golden rule}

  \begin{itemize}[<+- | alert@+->]

    \item You don't want to replicate data because
          \uncover<+->{you might need to change it in the future: if your data is
                  replicated all over, you need to change it all over (very
                  error-prone)}

    \item while if you keep relationships instead, a single change will be
            sufficient for each piece of datum you have

  \end{itemize}

\end{frame}

\refstepcounter{ms}
\begin{frame}
  \frametitle<+->{What are relations for? (\arabic{ms})}

  \begin{itemize}[<+- | alert@+->]

    \item This is what relations are for. For example:

    \begin{itemize}[<+- | alert@+->]

      \item If you have one \emph{or more} telephone numbers for each person
              in your agenda (which is likely, or make it e-mail addresses, or whatever)


      \item You need to be able to express something like \uncover<+->{``a
              person \emph{can have many} telphone numbers''}
              \uncover<+->{or like ``this number \emph{belongs to} this person''}

      \item Both sentences basically establish \emph{a one--way relation} between persons and
              phone numbers

    \end{itemize}
  \end{itemize}

\end{frame}

\setcounter{ms}{0}
\refstepcounter{ms}
\begin{frame}
 \frametitle<+->{How do you represent relationships inside computers?  (\arabic{ms})}

 \begin{columns}[T]
   \begin{column}{\leftcolumn}
     \uncover<3->{\scriptsize \verbatiminput{\exampledir/pim_ds.c}}
   \end{column}
   \begin{column}{\rightcolumn}
     \begin{itemize}[<+- | alert@+->]

        \item When your data is stored in RAM\dots

        \item \dots you represent relationships by \emph{memory pointers}
                (also known as \emph{references}) (check the arrow)

        \item but what happens when you want to \emph{persist}
                the relationship?
                (== save it to disk)

        \item this (and some more) is what relational databases are for

     \end{itemize}
   \end{column}
 \end{columns}
\end{frame}

\refstepcounter{ms}
\begin{frame}
  \frametitle<+->{How do you represent relationships inside computers?  (\arabic{ms})}

  \alert{Before we go on, a question for you.}

  \uncover<+->{
  \begin{center}
     \begin{large}
       \alert{How would you create a persistent telephone book\dots}\\\uncover<+->{\alert{with a filesystem?}}
     \end{large}
  \end{center}
  }

\end{frame}

\refstepcounter{ms}
\begin{frame}
  \frametitle<+->{How do you represent relationships inside computers?  (\arabic{ms})}

  \begin{itemize}[<+- | alert@+->]

    \item Current relational database can do much more than that

    \item Within databases, you

    \begin{itemize}[<+- | alert@+->]

      \item add a progressive number to each data structure

      \item this number is guaranteed (by the database engine) to be unique

      \item it is called the \emph{primary key}

    \end{itemize}

    \item Primary keys act like persistent pointers

  \end{itemize}

\end{frame}


\refstepcounter{ms}
\begin{frame}
  \frametitle<+->{How do you represent relationships inside computers?  (\arabic{ms})}

  \begin{itemize}[<+- | alert@+->]

    \item Relationships are expressed with further indexes, called
            \emph{foreign keys}

    \item Theoreticians have established that with only \emph{three}
            kinds of relationships you can describe any type of relation among
            data

    \item These relationships are\dots

  \end{itemize}

\end{frame}

\pgfdeclareimage[height=0.5\textheight]{model relationship}{\imagedir/ER}

\refstepcounter{ms}
\begin{frame}
  \frametitle<+->{How do you represent relationships inside computers?  (\arabic{ms})}

  \begin{itemize}[<+- | alert@+->]

    \item one--to--one \uncover<+->{\normalsize (f.ex: \emph{a husband has only one wife})}

    \item one--to--many \uncover<+->{\normalsize (f.ex: \emph{a football team has many players})}

    \item many--to--many \uncover<+->{\normalsize (f.ex: \emph{a student has many subjects})}

  \end{itemize}
  \only<+->{\centering \pgfuseimage{model relationship}}

\end{frame}

\pgfdeclareimage[width=\leftcolumn]{one to one}{\imagedir/OneToOne}

\begin{frame}
  \frametitle<+->{One--to--one relationships}

  \begin{columns}[T]
    \begin{column}{\leftcolumn}
      \only<5->{\pgfuseimage{one to one}}
    \end{column}
    \begin{column}{\rightcolumn}
      \begin{itemize}[<+- | alert@+->]
    
        \item you establish one--to--one relationships by:


           \begin{itemize}[<+- | alert@+->]

              \item creating a first table with an auto--generated primary key

              \item creating a second table with a (\emph{not}
                      auto--generated) foreign key which points to an instance
                      of the first table
    
           \end{itemize}

      \end{itemize}
    \end{column}
  \end{columns}

\end{frame}

\pgfdeclareimage[width=\leftcolumn]{one to many}{\imagedir/ManyToOne}

\begin{frame}
  \frametitle<+->{One--to--many relationships}

  \begin{columns}[T]
    \begin{column}{\leftcolumn}
      \only<5->{\pgfuseimage{one to many}}
    \end{column}
    \begin{column}{\rightcolumn}
      \begin{itemize}[<+- | alert@+->]
    
        \item you establish one--to--many relationships by:


           \begin{itemize}[<+- | alert@+->]

              \item creating a first table with an auto--generated primary key

              \item creating a second table with an auto--generated primary key
                      \emph{and} a foreign key which points to an instance
                      of the first table
    
           \end{itemize}

      \end{itemize}
    \end{column}
  \end{columns}

\end{frame}

\pgfdeclareimage[height=0.5\textheight]{many to many}{\imagedir/ManyToMany}

\begin{frame}
  \frametitle<+->{Many--to--many relationships}

    \begin{itemize}[<+- | alert@+->]
  
      \item you establish many--to--many relationships by:


         \begin{itemize}[<+- | alert@+->]

              \item creating two separate tables with an auto--generated primary keys

              \item creating \emph{a third} table with \emph{two} foreign keys
                    each pointing to one instance of one of the first two tables
  
         \end{itemize}

    \end{itemize}
    \begin{center}
    \only<+->{\pgfuseimage{many to many}}
    \end{center}

\end{frame}

\begin{frame}
  \frametitle<+->{A real--world example}

  \begin{center}
    \pgfimage[height=0.95\textheight]{\imagedir/shop_db_schema}
  \end{center}

\end{frame}

\section[SQL]{Query Language}

\setcounter{ms}{0}
\refstepcounter{ms}
\begin{frame}
  \frametitle<+->{A structured query language (SQL) (\arabic{ms})}

  \begin{itemize}[<+- | alert@+->]

    \item Of course, structuring data is not enough to be able to use it
          efficiently

    \item You also need to have a way to create, browse, view, update, delete
          data 

    \item Such a language exists: it is called \emph{SQL} (spelled:
            \emph{sequel} -- which stands precisely for \emph{Structured Query
            Language})

    \item With minor changes, SQL can be used over a multitude of different
            databases ({\tt sqlite}, {\tt MySql}, {\tt PostgreSQL}, {\tt Oracle}, {\tt Informix}, etc.)

  \end{itemize}

\end{frame}

\refstepcounter{ms}
\begin{frame}
  \frametitle<+->{A structured query language (SQL) (\arabic{ms})}

  \begin{itemize}[<+- | alert@+->]

    \item SQL is a text--based language: you type some instruction on a database
            console and the database will reply with an answer

    \item Of course it can also be scripted and put into a file

    \item SQL has several commonly used statements:

    \item ``SELECT'', \uncover<+->{``INSERT'', }\uncover<+->{``UPDATE'', }
            \uncover<+->{``DELETE'', }\uncover<+->{``CREATE''}
            \uncover<+->{ and ``DROP''}

    \item ``SELECT'' is the most commonly used, as it is the statement that
            allows you to make queries (it does not change the database)

  \end{itemize}

\end{frame}

\subsection{SELECT}

\setcounter{ms}{0}
\refstepcounter{ms}
\begin{frame}
  \frametitle<+->{The ``SELECT'' statement (\arabic{ms})}

  \alert{Here is what the ``SELECT'' statement looks like:}

  \begin{center}
  \begin{large}
    \uncover<+- | alert@+->{\verbatiminput{\exampledir/select.sql}}
  \end{large}
  \end{center}

\end{frame}

\refstepcounter{ms}
\begin{frame}
  \frametitle<+->{The ``SELECT'' statement (\arabic{ms})}

  \begin{itemize}[<+- | alert@+->]

    \item Some ``SELECT'' examples:

     \begin{itemize}[<+- | alert@+->]

       \item {\tt select first, last, city from employees where first like 'Er\%';}
               \uncover<+->{find all first names that begin with `Er' and
               display first name, last name and city of operation}

       \item {\tt select * from employees where first = 'Eric';}
               \uncover<+->{find all first names that match the name `Eric' exactly and display all data from them}

       \item {\tt select last, city, age from employees where age > 30;}
               \uncover<+->{find all employees which are older than 30 years
               of age and display last name, city and age}

       \item {\tt select last, city, age from employees where (age > 30) and (last like '\%s');}
               \uncover<+->{find all employees which are older than 30 years
               of age \emph{and} have a last name that ends in `s' and display last name, city and age}

        \item \dots \uncover<+->{let's do some examples live\dots}

     \end{itemize}

  \end{itemize}

\end{frame}

\section{Exercising}

\begin{frame}
  \frametitle<+->{Live examples}

  \begin{itemize}

    \item Please download the employees.sqlite from Moodle

    \item If you have a Mac or a Linux laptop, you may open a terminal and
            type {\tt sqlite3 employees.sqlite} in the folder where you
            downloaded the db

    \item if you have Windows, you may find a pre--compiled binary at \url{https://www.sqlite.org/download.html}

  \end{itemize}

\end{frame}

\begin{frame}
  \frametitle<+->{Walk-through exercise}

  \begin{itemize}[<+- | alert@+->]

    \item Let's imagine reverse--engineering\dots \uncover<+->{FaceBook}
            \uncover<+->{at least partially :-)}

  \end{itemize}

\end{frame}

\begin{frame}
  \frametitle<+->{DIY Exercise (30 minutes)}

  \begin{enumerate}[<+- | alert@+->]

    \item Pick your own data set and/or application

    \item Design it in terms of a relational database:

     \begin{enumerate}[<+- | alert@+->]

       \item Establish tables

       \item Establish relationships among tables, making sure that no data is
             replicated

     \end{enumerate}

     \item Discuss your design

  \end{enumerate}

\end{frame}

  %
  % backup plan for standalone exercise: Rejsekort
  %

\section{Wrap Up}

\begin{frame}
	\frametitle<+->{What have we covered}

  \begin{enumerate}[<+- | alert@+->]

    \item What are databases and why do they exist at all

    \item How they are conceived

		\item What are data relationships

		\item An introduction to the SQL language

		\item How to design proper database structures (with some exercising)

  \end{enumerate}

\end{frame}

\begin{frame}

	\begin{center}
	\begin{Huge}
		\alert{THANK\ YOU!}
	\end{Huge}
	\end{center}

\end{frame}

\end{document}
